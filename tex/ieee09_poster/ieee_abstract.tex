%
% Use 'make' to run latex and generate the pdf file
%
% Doug Wright
%
% ADC 
%
% COK-2001-600 
% 11-304 Physics concepts such as hydrodynamics, photon transport,
% neutronics, fission, fusion, etc., when no classified information
% or association is revealed.

\documentclass[fleqn,11pt]{article}
\usepackage{h4}

\newcommand{\notgeant}[1]{#1} % include stuff that is not for geant manual

\date{May 11, 2009}

\version{}   % doc version number
\ucrl{LLNL-ABS-413079}

\mytitle{Detailed photofission physics library for Monte-Carlo radiation transport codes\footnote{This work was performed under the auspices of the U.S. Department of Energy by Lawrence Livermore National Laboratory under Contract DE-AC52-07NA27344.}
\author{
%%Fission Library Development Team\\
Jerome M. Verbeke\footnote{Contact info: verbeke2@llnl.gov, 925-422-8337}, Doug Wright\\}
\\
Lawrence Livermore National Laboratory
}

\begin{document}
\maketitle

% \tableofcontents
% \clearpage

% \section{Photonuclear physics}

Detection of nuclear weapons and special nuclear materials (SNM, certain types of
uranium and plutonium) is crucial to thwarting nuclear proliferation and terrorism
and to securing weapons and materials worldwide. Congress has funded a portfolio
of detection R\&D and acquisition programs, and has mandated inspection at foreign
ports of all U.S.-bound cargo containers.

Nuclear weapons contain SNM, which produce unique or suspect signatures that
can be detected. For instance, since they are very dense, they can be seen by 
x-rays in the absence of heavy shielding. One class of SNM signatures detectable 
passively is the emission of neutrons and gamma-rays due to both natural 
radioactivity of actinides, and cosmic-ray induced fission. Passive detection 
systems rely on a variety of such signatures ranging from spectroscopic 
information from the decay of actinides, all the way to the detailed time 
structure of the fission chains in the multiplying SNM. These signatures can be 
enhanced by inducing fissions in the SNM using a neutron or photon beam in an 
active interrogation detection scheme.

National labs and private corporations are presently developing detection 
technologies based on prompt and delayed fission gamma-rays and neutrons. 
Current physics modeling capabilities of established Monte-Carlo radiation 
transport codes are unfortunately limited and often do not offer the detailed 
fission physiscs modeling required by both passive and active detection systems.

The fission physics library developed at Lawrence Livermore National Lab aims
at filling that need of the simulation community by offering detailed data-based 
fission physics. The data-driven library models spontaneous, neutron-induced 
fission as well as photofission. Since mcnpx v27b, the library can be turned on
transparently via a programmatic interface.

% \clearpage
% \addcontentsline{toc}{section}{References}

% \begin{thebibliography}{99}
% \bibitem{Oblozinsky 1998} "Summary Report of the $2^{nd}$ Research Coordination 
Meeting on Compilation and Evaluation of Photonuclear Data for 
Applications," International Atomic Energy Agency Report INDC(NDS)-384, 
Vienna, Austria (1998).

\bibitem{McLane 1997} V. McLane, C.L. Dunford and P.F. Rose, "ENDF-102 Data 
Formats and Procedures for the Evaluated Nuclear Data File ENDF-6," Report 
BNL-NCS-44945 Rev. 2/97, Brookhaven National Laboratory, Upton, New York 
(1997).

\bibitem{Chadwick 1999} M.B. Chadwick et al., "Cross-Section Evaluations to 
150 MeV for Accelerator-Driven Systems and Implementation in MCNPX," 
\textit{Nuclear Science and Engineering,} \textbf{131,} pp. 293-328 (1999).

\bibitem{Mashnik 2001} S.G. Mashnik and A.J. Sierk, "Recent Developments
of the Cascade-Exciton Model of Nuclear Reactions," Los Alamos National 
Laboratory report LA-UR-01-5390, and International Conference on Nuclear 
Data for Science and Technology, Tsukuba, Japan, October 7-12, 2001.

\bibitem{Mashnik 2008} S.G. Mashnik, K.K. Gudima, R.E. Prael, A.J. Sierk, 
M.I. Baznat, and N.V. Mokhov, "CEM03.03 and LAQGSM03.03 Event Generators
for the MCNP6, MCNPX, and MARS15 Transport Codes," Los Alamos National
Laboratory report LA-UR-08-2931, and invited lectures at the Joint
ICTP-IAEA Advanced Workshop on Model Codes for Spallation Reactions,
Trieste, Italy, February 4-8, 2008.

% \end{thebibliography}

\end{document}
