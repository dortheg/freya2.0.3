\pagebreak
\subsection{COG}
This physics module has been submitted to the COG development team
and will appear in a future release.  The fission library libFission.a
can be sampled for induced fissions in COG using the \textit{FISSLIB}
keyword in the MIX block of the input deck. This is not the
default. The \textit{NUOPTION} can not be used concurrently, it is not
compatible with \textit{FISSLIB}.

COG is similar to MCNPX in that it emits a number of gamma-rays
at each neutron collision site, and this number is independent
on the reaction type. The keyword \textit{NOGAMPRO} can be used
to completely turn off this gamma-ray production.

Spontaneous fissions are implemented using a COG user source.
A sample user source {\tt spfiss.F} is available in the COG distribution 
subdirectory {\tt usrsor}. Compiling a COG user source is easy:
\begin{verbatim}
	make -f COGUserlib.make in=spfiss.F
\end{verbatim}
Using the right compiler at compile time is important, and if this
becomes an issue, a COG developer should be contacted. It is also 
important to use a COG version that is compatible with user
sources and user detectors. Not all COG versions work with
user sources/detectors. Photofission has not been implemented in
COG.

An example of input deck using the {\tt spfiss.F} spontaneous
fission source is located in the {\tt usrsor} directory. The important
lines related to the user source are in the SOURCE block:

{ \small
\begin{verbatim}
SOURCE
NPART = 5e4  $ NPART is the sum of spontaneous fission neutrons and photons
$ 
$ The source below is for a HEU shell (93% enriched in U-235).
$ We neglect here the spontaneous fissions in U-235. The fission rate 
$ for 350 g of U-238 is 350[g]*1.36*10E-2[n/g/s] = 4.76 n/s. With
$ spontaneous nubar equal to 2.01, we have 
$  4.76[n/s]/2.01[n/fission] = 2.368 fissions/sec
$      name    isotope  strength  xcenter  ycenter  zcenter  Rin  Rout  FissRate
$              (1)      (2)       (3)      (4)      (5)      (6)  (7)   (8)
USRSOR spfiss  92238    1.        0.       0.       0.       1.   3.96  2.368
$
\end{verbatim}
}
NPART is the sum of all source particles, that is both spontaneous
fission neutrons and gamma-rays. The line USRSOR has several arguments:
The argument under {\tt name} specify the FORTRAN subroutine to be used a 
the spontaneous fission source: {\tt spfiss}. The first numeric argument is 
the isotope in the form ZA, followed by the source strength (not relevant
in this case), the center of the shell (x, y, z), the inner and outer
radii and the fission rate in fissions/second. Note the units of the
fission rate, fissions/second and not neutrons/second.

