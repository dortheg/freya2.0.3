
For neutron induced fission, this model is intended to be used with
the low energy neutron interaction data libraries with class
\textit{G4Fisslib} specified in the physics list as the
\textit{G4HadronFissionProccess} instead of class
\textit{G4NeutronHPFission}.\notgeant{
Here is an example code snippet for registering this model in the physics 
list: \begin{verbatim}
    G4ProcessManager* pmanager = particle->GetProcessManager();
    G4String particleName = particle->GetParticleName();

    if (particleName == "gamma") {
      (...)
    } else if (particleName == "neutron") {
      (...)
      // Fission library model
      G4HadronFissionProcess *theFissionProcess = new G4HadronFissionProcess();
      G4FissLib* theFissionModel = new G4FissLib;
      theFissionProcess->RegisterMe(theFissionModel);
      pmanager->AddDiscreteProcess(theFissionProcess);
      (...)
    } else ...
\end{verbatim}
}
The constructor of \textit{G4FissLib}
does two things. First it reads the necessary fission cross-section
data in the file located in the directory specified by the environment
variable \textit{NeutronHPCrossSections}. It does this by initializing
one object of class \textit{G4NeutronHPChannel} per isotope present in
the geometry. Second, it registers an instance of
\textit{G4FissionLibrary} for each isotope as the model for that
reaction/channel. When Geant4 tracks a neutron to a reaction site and
the fission library process is selected among all other process for
neutron reactions, the method \textit{G4FissLib::ApplyYourself} is
called, and one of the fissionable isotopes present at the reaction
site is selected. This method in turn calls
\textit{G4NeutronHPChannel::ApplyYourself} which calls
\textit{G4FissionLibrary::ApplyYourself}, where the induced neutrons
and gamma-rays are emitted by sampling the fission library.

For spontaneous fission the user must provide classes {\it
PrimaryGeneratorAction}, {\it MultipleSource}, {\it
MultipleSourceMessenger}, {\it SingleSource}, {\it SponFissIsotope} to
generate spontaneous fission neutrons and gammas. Examples of these
classes can be downloaded from \httpnuclear. Spontaneous fissions are
generated in the {\it PrimaryGeneratorAction} class.
The spontaneous fission
source needs to be described in terms of geometry, isotopic
composition and fission strength. Once this information is given, the
constructor creates as many spontaneous fission isotopes of class {\it
SponFissIsotope} as specified, and adds them to the source of class
{\it MultipleSource}. When Geant needs to generate particles, it calls
the method {\it PrimaryGeneratorAction::GeneratePrimaries}, which
first sets the time of the next fission based on the fission rates
entered in the constructor, and then calls the method {\it
MultipleSource::GeneratePrimaryVertex} which determines which one of
the spontaneous fission isotopes will fission. This method in turn
calls the method {\it SponFissIsotope::GeneratePrimaryVertex} for the
chosen isotope. It is in this method that the neutrons and photons
sampled from the fission library are added to the stack of secondary
particles.  Sources other than spontaneous fission isotopes can be
added to the source of class {\it MultipleSource}. For instance, a
background term emitting a large number of background gamma-rays can
be added, as long as it derives from the class {\it SingleSource}. The
intensity of that source would be set the same way as for the
spontaneous fission isotope sources.
