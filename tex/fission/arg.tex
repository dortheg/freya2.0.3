% !TEX root =  fission.tex
\subsection*{void setnudist\_(int *nudist)
\label{setnudist}}

This selects the data to be sampled for the neutron number distributions for neutron-induced fission. If there is no data
available, then in all cases the Terrell approximation is used. The argument \textit{nudist} can take 3 values:

\begin{tabbing}

\indent 0 \hspace*{.55in} \= \parbox[t]{5.5in}{ Use the fit to the Zucker and Holden tabulated P$_\nu$ distributions as a function of energy for $^{235}$U, $^{238}$U and $^{239}$Pu.}\\

\indent 1 \> \parbox[t]{5.5in}{Use fits to the Zucker and Holden tabulated P$_\nu$  distribution as a function of energy for $^{238}$U and  $^{239}$Pu, and a fit to the Zucker and Holden data as well as the Gwin, Spencer and Ingle data (at thermal 
 energies) as a function of energy for $^{235}$U.}\\

\indent 2 \> \parbox[t]{5.5in}{Use the fit to the Zucker and Holden tabulated P$_\nu$ distributions as a function of $\bar{\nu}$. The $^{238}$U fit is used for the $^{232}$U, $^{234}$U, $^{236}$U and $^{238}$U isotopes, the $^{235}$U fit for $^{233}$U 
and $^{235}$U, the $^{239}$Pu fit for $^{239}$Pu and $^{241}$Pu.}\\

\indent 3 (default) \> \parbox[t]{5.5in}{Use the discrete Zucker and Holden tabulated P$_\nu$ distributions and corresponding $\bar{\nu}$s. Sampling based on the incident neutron $\bar{\nu}$. The $^{238}$U data tables are used for the $^{232}$U, $^{234}$U, $^{236}$U  and $^{238}$U isotopes, the $^{235}$U data for $^{233}$U and $^{235}$U, the $^{239}$Pu data for $^{239}$Pu and $^{241}$Pu.}

\end{tabbing}

\subsection*{void setcf252\_(int *ndist, int *neng)}

This function is specific to the spontaneous fission of $^{252}$Cf. It selects the data to be sampled for the neutron number and energy distributions and takes the following arguments:

\begin{tabbing}
\indent ndist: \= Sample the number of neutrons \\
\indent \> 0 (default) \= 
from the tabulated data measured by Spencer \\
\indent \> 1 \> from 
Boldeman's data \\
\\
\indent neng: Sample the spontaneous fission 
neutron energy \\
\indent \> 0 (default)\> from Mannhart corrected  Maxwellian spectrum \\
\indent \> 1 \> from Madland-Nix theoretical spectrum \\
\indent \> 2 \> from the Froehner Watt spectrum \\
\end{tabbing}

\subsection*{void getfreya\_errors\_(int *length, char *error)}
When called, this function returns potential errors that could have occurred in {\tt FREYA}; for instance if the data required by {\tt FREYA} cannot be found. It takes the following arguments:
\begin{tabbing}
\indent length: \= length of error message \\
\indent error: \> pointer to an allocated array of characters devoted to the error message \\
\end{tabbing}
When returning, the length of the error message will be 1 is no error occurred. It will be greater than 1 otherwise.

\subsection*{void setfreyadatapath\_(char *path)}
This function is called to set the path to the directory where the data required by {\tt FREYA} is located. It takes the following argument:
\begin{tabbing}
\indent path: \= character string containing the path to the directory containing the data required by {\tt FREYA}. \\
\end{tabbing}

