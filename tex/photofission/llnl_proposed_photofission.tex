%
% pdflatex StateOfTheArt.tex
%
% Doug Wright
%
% ADC 
%
% COK-2001-600 
% 11-304 Physics concepts such as hydrodynamics, photon transport,
% neutronics, fission, fusion, etc., when no classified information or association is revealed.
%
\documentclass[fleqn,11pt]{article}
\usepackage{h4}

\newcommand{\notgeant}[1]{#1} % include stuff that is not for geant manual

\date{October 24, 2008}

\version{}   % doc version number
\ucrl{UCRL-AR-}

\mytitle{LLNL Proposed Modifications to the Treatment of Photofission in MCNPX}

\author{
Jerome M. Verbeke, Doug Wright%
\footnote{Contact info: wright20@llnl.gov, 925-423-2347}\\
\\
Lawrence Livermore National Laboratory
}

\begin{document}
\maketitle

%\tableofcontents\clearpage

%%%%%%%%%%%%%%%%%%%%%%%%%%%%%%%%%%%%%%%
\section{MCNPX photon collision sampling}
In MCNPX, all of the photonuclear reactions are grouped together, so that whenever
a photon undergoes such a reaction, the number of emitted particles (and their energy and angle)
are sampled from a single distribution made by averaging over all reactions. This is obviously not correct on an event-by-event basis. Furthermore, the
mean number of particles produced is not done with analog sampling, but instead is either $n$ or $n+1$ in order
to give the measured average. 

MCNPX determines the distance to the next photon collision based 
on the total photon cross-section computed from the sum of 
photoatomic (incoherent, coherent, photoelectric, pair 
production) and photonuclear (nonelastic (gamma,n), elastic, 
photofission (gamma,f), ...) cross-sections using the material 
composition and density. Then it decides whether a photonuclear 
or photoatomic interaction occured in the subroutine \textit{collpn}, 
and then with which nucleus. The mean number of secondary 
particles produced for each type (e.g., neutron, photon, 
proton, ...) is extracted from external data (ratio of 
production cross-section to the total cross-section). This 
is sampled such that n or n+1 particles are produced to 
give the correct mean number.
For each available particle type the secondaries are assigned to
sub-processes, e.g. 3 photons from photofission and 2 photons from
some other photonuclear reaction (gamma,n). Thus multiple processes
are happening at the same time, which is non-physical.

Libraries are used when available, and models are used otherwise.
MCNPX has a mix-and-match capability enabling mixing and matching
of physics models and data tables. It is possible to specify some
nuclides with models and other nuclides with data tables (isotope
``mixing"). It is also possible to use data tables up to their
maximum energy value and then use models above that energy, even
when maximum table energy differs from nuclide to nuclide
(energy ``matching"). Photonuclear physics is modeled with the new
CEM model~\cite{Mashnik 2001}~\cite{Mashnik 2008}, regardless of 
whether CEM is used for other particles.

\section{Proposed photofission treatment}
In this project, we are implementing an alternative method for 
photonuclear reactions. The default is still the original 
method as described above, but the new alternative method can 
be turned on by the new `illnlphfis' switch ($7^{th}$ entry of 
the photon physics card PHYS:P). It also requires the photonuclear
reaction treatment to be set to analog by setting the `ispn' 
switch ($4^{th}$ entry of the photon physics card PHYS:P) to -1. 
With the new method, the first
2 steps are identical to the ones for the default method: The 
code first determines whether to follow the photoatomic or 
the photonuclear path and then picks a nuclide with which to
interact. For that nuclide, the code computes the probability 
of photofission, which is the ratio of photofission 
cross-section to total cross-section, themselves read from 
data libraries. 
The code then determines probabilistically if a photofission 
occurs. If photofission is the chosen photonuclear reaction,
mcnpx banks the neutrons and gammas generated by the LLNL 
photofission library. The numbers of neutrons and gammas, 
as well as their respective energies, are all coming from 
this single photofission library. On the other side, if we 
determine that a different photonuclear took place in lieu 
of photofission, we revert to the original treatment in the 
\textit{collpn} routine. However, in this case, 
to avoid counting twice particles emitted by photofission
(reaction which was already accounted for in the new part of 
the code), we also drop any particle coming from the 
photofission process from the ``reverted-to" default method.

In other words, this new `illnlphfis' option on the PHYS:P
card allows the user to simulate the photofission process
exactly, and therefore enables coincidence counting of 
photofission neutrons and gammas for instance. This is
different from the default, where secondary particles
emitted by photonuclear interactions are only correct
on average over a large number of interactions, because
the number of secondary particles, as well as their 
energies and directions are averaged over all possible 
photonuclear interactions.

Regardless of the option chosen in the $7^{th}$ entry
of the PHYS:P card, delayed neutrons and gammas from
photofission can be turned on and off independently 
via a different switch.

\section{Proposed new entries to the mcnpx summary tables}

The $7^{th}$ entry (illnlphfis) on the PHYS:P card 
enables the user to count exactly how many photofission 
interactions occured versus other photonuclear 
interactions. Therefore, we can now report the number 
of these photofission interactions in the summary 
tables as well as in the different 130 tables. 

\subsection{Changes to the summary table for neutrons}
In the neutron creation column, a `prompt photofission'
entry was added that counts how many prompt neutrons are
created from photofission. The `photonuclear' entry that
used to count neutrons from all photonuclear processes
(including prompt neutrons from photofission) will no 
longer include the prompt neutrons from photofission, 
which are tallied separately in `prompt photofission'.

\subsection{Changes to the summary table for photons}
In the photon creation column, a `prompt photofission'
entry was added that counts how many prompt photons are 
created from photofission. The `photonuclear' entry that
used to count photons from all photonuclear processes
(including prompt photons from photofission) will no 
longer include the prompt photons from photofission, 
which are tallied separately in `prompt photofission'.

In the photon loss column, the `photonuclear abs' entry
that used to count the photons absorbed in all 
photonuclear processes (including photofission), will
no longer count the photons absorbed in photofission,
which will be tallied separately in the `loss to 
photofission' entry. The `loss to photofission' entry 
in the photon loss column will be zero unless the 
7$^{th}$ entry (illnlphfis) of the PHYS:P card is set 
to 1 and the 4$^{th}$ entry (ispn) is set to -1. This 
is because the original setting (illnlphfis=0) 
does not generate secondary particles adequately on a 
reaction per reaction basis. It only works correctly 
in average over a large number of reactions. When 
illnlphfis=0, the photons absorbed by photofission 
reactions are counted towards the `photonuclear abs' 
entry.

\subsection{Changes to table 130 for neutrons (physical events)}
The `photofission' entry was added to count the neutrons 
created by photofission. These used to be counted 
together with the other neutrons created by all other 
photonuclear processes in the 'photonuclear' entry.

\subsection{Changes to table 130 for photons (physical events)}
The `photofission' entry was added to count the photons 
created by photofission. These used to be counted 
together with the other photons created by all other
photonuclear processes in the `photonuclear' entry.

The `loss to photofission' entry was added to the
physical events for photons. Photons lost to
photofission will be counted in that entry, instead
of being counted towards the `photonuclear abs' entry.
`lost to photofission' is populated only when
illnlphfis=1. If illnlphfis=0, photons lost to 
photofission are counted towards the `photonuclear 
abs' entry.

\addcontentsline{toc}{section}{References}

\begin{thebibliography}{99}

%\bibitem{Oblozinsky 1998} "Summary Report of the $2^{nd}$ Research Coordination 
%Meeting on Compilation and Evaluation of Photonuclear Data for 
%Applications," International Atomic Energy Agency Report INDC(NDS)-384, 
%Vienna, Austria (1998).
%
%\bibitem{McLane 1997} V. McLane, C.L. Dunford and P.F. Rose, "ENDF-102 Data 
%Formats and Procedures for the Evaluated Nuclear Data File ENDF-6," Report 
%BNL-NCS-44945 Rev. 2/97, Brookhaven National Laboratory, Upton, New York 
%(1997).
%
%\bibitem{Chadwick 1999} M.B. Chadwick et al., "Cross-Section Evaluations to 
%150 MeV for Accelerator-Driven Systems and Implementation in MCNPX," 
%\textit{Nuclear Science and Engineering,} \textbf{131,} pp. 293-328 (1999).

\bibitem{Mashnik 2001} S.G. Mashnik and A.J. Sierk, ``Recent Developments
of the Cascade-Exciton Model of Nuclear Reactions," Los Alamos National 
Laboratory report LA-UR-01-5390, and International Conference on Nuclear 
Data for Science and Technology, Tsukuba, Japan, October 7-12, 2001.

\bibitem{Mashnik 2008} S.G. Mashnik, K.K. Gudima, R.E. Prael, A.J. Sierk, 
M.I. Baznat, and N.V. Mokhov, ``CEM03.03 and LAQGSM03.03 Event Generators
for the MCNP6, MCNPX, and MARS15 Transport Codes," Los Alamos National
Laboratory report LA-UR-08-2931, and invited lectures at the Joint
ICTP-IAEA Advanced Workshop on Model Codes for Spallation Reactions,
Trieste, Italy, February 4-8, 2008.

\end{thebibliography}

\end{document}
