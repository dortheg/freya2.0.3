FREYA (Fission reaction event yield algorithm) is a code that simulates the decay of a fissionable nucleus at a specified excitation energy. In its present form, FREYA ignores the possibility of neutron emission from the nuclear prior to its fission ($n^{th}$ chance fission), and its applications are therefore limited to lower energies, such as thermal fission. 

This short paragraph briefly summarizes the physics included inf FREYA. When the possibility of prefission radiation is ignored, the first physics issues concern how the mass and charge of the initial compoung nucleus is partitioned among the two fission fragments and how the available energy is divided between the excitation of the two fragments and their kinetic energy. Because there is no good theory for describing the gradual deformation of the initial compound nucleus, the masses and charges of the fission fragments are selected from probability distributions. The two excited fragments are assumed not to de-excite until after they have been fully accelerated by their mutual Coulomb repulsion and their shapes have reverted to their equilibrium form, which are taken to be their ground states. Emission of charged particles from the fission fragments is ignored. Each of the fission fragments emits multiple neutrons until neutron emission becomes energetically impossible. It is assumed that neutron evaporation is complete before photon emission sets in. The fission fragments then typically emit a number of photons that is larger than the number of neutrons. At this initial stage, the main focus of FREYA has been the neutron evaporation, but it includes an approximate treatment of photon emission: nuclear structure effects are ignored and the postevaporation photon cascade is treated in a manner that is similar to the neutron emission. 

FREYA has currently a few limitations: It only works for first chance fission, even though multiple chance fission is being added. It has only being tested, tuned and benchmarked against experimental data for $^{239}$Pu and $^{235}$U. Spontaneous fission and photofission are currently not implemenented. FREYA could be used to determine the number correlation between neutrons and photons emitted by fission as well and their energies, but only for low energy neutron inducing fission on $^{239}$Pu and $^{235}$U. Because of these limitations and the scope of our project, this option was not considered further.
